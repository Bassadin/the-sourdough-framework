In this chapter you will learn about different
methods of storing your bread. This way
your bread can be best enjoyed at a later
time.

\begin{table}[!htb]
    \begin{center}
        \begin{tabular}{@{}>{\bfseries}p{0.3\textwidth}p{0.3\textwidth}p{0.3\textwidth}@{}}
\toprule
\thead{Method}     & \thead{Advantages}            & \thead{Disadvantages}       \\ \midrule
Room temperature              & The easiest option. Best for bread that is eaten within a day.
                                Crust typically stays crisp when humidity not too high.
                              & Bread dries out very quickly.\\ 

Room temperature in container & Good for up to a week. Catches mold more quickly.
                              & Bread needs to be toasted for crust to become crisp again.\\ 

Fridge                        & Bread stays good for weeks. Can dry out a little bit when not using air-tight container.
                              & Bread needs to be toasted. Requires fridge and energy.\\ 

Freezer                       & Bread stays good for years.
                              & Requires thawing and then toasting. Requires freezer and energy.\\

\bottomrule
\end{tabular}

        \caption{A table visualizing the advantages and disadvantages
        of different bread storing options.}
        \label{table:bread-storage}
    \end{center}
\end{table}

\section{Room temperature}

The most common method is to store your bread
at room temperature. After taking a slice of bread,
store your bread with the crumb facing side
downwards.

This method works great if you want to eat
your bread within a day. The crust stays
crisp and does not become soft. \footnote{
  The higher the humidity in your room, the faster
  the crust will become soft.
}. The biggest downside to this method is that
the bread becomes hard quickly. As time progresses,
more and more water evaporates from your dough's
crumb. Ultimately, the bread will become very hard
and impossible to eat. The more water you use
to make the bread, the longer the bread stays good.
A low-hydration recipe can dry out after 1-2 days;
a high-hydration bread needs 3-4 days to dry out.

Once your bread has dried out, you can run it under
tap water for around 10 to 15 seconds. 
This water bath allows the
crumb's starch to absorb a lot of water. Proceed and
bake your bread again in the oven. The resulting loaf
will be almost as good as new again.

Another option for dried-out bread is to use it
to make breadcrumbs. These bread crumbs can be mixed
into subsequent loaves. They can also be used as
base ingredients for other recipes such as "Knödel".\footnote{
  Knödel is an Austrian dish that uses old bread as a basis.
  Breadcrumbs and day-old bread are mixed with eggs, and sometimes
  spinach or ham are added. The batter is then boiled in salty water.
}

\section{Room temperature in a container}

Just like the previous option, you can also store your
bread inside a container. This could be a paper bag, 
a plastic bag, or a bread storage box. The paper bag and
most bread boxes are not fully sealed. They allow some of
the air to diffuse out of the container. This means that
the bread will also slightly dry out.

When using a sealed bag such as a plastic bag, the bread
will retain a lot of moisture. The bread will stay good
for a longer period. However, at the same time, the crust
will also lose its crispness. Some of the water diffuses
into the bag and is then re-absorbed by the crust. If
you want the crisp crust, the best option is to toast your
bread.

Another problem with storage containers is natural
mold contamination. The moment your bread is taken out of
the oven it starts being contaminated with aerial mold spores.
The spores are microscopically small and are everywhere.
The mold spores grow best in a humid environment. By placing
your dough in a container you have created a mold paradise.
A plain yeast-based dough will start to mold within a few days
like this. The sourdough-based bread stays good
for a longer period as the acidity is a natural mold
inhibitor.

\section{Fridge}

In my own experience storing bread inside the fridge
works well as long as you use a sealed container. Some
sources say that the bread dries out inside of the
fridge \cite{storing+bread}. Supposedly the fridge
encourages liquid from the crumb to migrate to the bread's surface.

In my experience though, the trick is to use a sealable
container. With a sealable ziplock bag,
the excess humidity will stay in the bag and ensures
that the bread does not dry out as quickly. At room
temperature, this would cause your bread to mold. At
lower temperatures, the bread can stay good like this for
weeks. The crust however, will lose its crispness and
thus toasting is advised.

\section{Freezing}

Another great option for long-term storage is to use
your freezer. Slice up the whole loaf and create portions
that you can consume within a day. Store each portion
in a separate container and place them inside your
freezer.

When you want to eat fresh bread, open one of the portions
in the morning and allow the bread to thaw over a few
hours. This way you can easily remove the frozen-together
slices. Proceed and toast the slices in your toaster
or bake them in the oven until they have the crispness
that you like.

This option is great for very long-term storage. Personally
I like having a few slices of bread frozen as an emergency
backup when I have had no time to bake.
