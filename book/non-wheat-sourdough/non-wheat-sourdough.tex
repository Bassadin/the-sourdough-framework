\chapter{Non wheat sourdough}%
\label{chapter:non-wheat-sourdough}
\begin{quoting}
In this chapter you will learn how to make a basic sourdough bread
using non-wheat flour, basically all flour except spelt.
The key difference between wheat and non-wheat flour is
the quantity of gluten, the former feature a high amount
of gluten, while the non-wheat flours do not.
\end{quoting}

The whole process (see Flowchart~\ref{flc:non-wheat-sourdough}) is a lot
easier: you mix the ingredients and wait for a certain period until the dough
has reached the level of acidity that you like.  Afterward, you shape the
dough or pour it into a loaf pan. After a short proofing period, the bread can
be baked. Due to the lack of gluten development, the final bread will feature
a denser crumb compared to wheat, as you can see in
Picture~\ref{fig:rye-crumb}.

\begin{flowchart}[!htb]
\begin{center}
  \begin{tikzpicture}[node distance = 3cm, auto]
  \node [start] (init) {\footnotesize Mix ingredients};
  \node [block, below of=init, node distance=3cm] (bulk_ferment) {\footnotesize Bulk ferment};
  \node [block, right of=init, node distance=3cm] (divide) {\footnotesize Divide};
  \node [block, below of=divide, node distance=3cm] (shape) {\footnotesize Shape};
  \node [block, right of=divide, node distance=3cm] (proof) {\footnotesize Proof};
  \node [success, below of=proof, node distance=3cm] (bake) {\footnotesize Bake};
  \path [line] (init) -- (bulk_ferment);
  \path [line] (bulk_ferment) -- (divide);
  \path [line] (divide) -- (shape);
  \path [line] (shape) -- (proof);
  \path [line] (proof) -- (bake);
\end{tikzpicture}

  \caption[Process for non-wheat sourdough bread]{A visualization of the
      process to make non-wheat sourdough bread.  The process is much simpler
      than making wheat sourdough bread. There is no gluten development. The
      ingredients are simply mixed together.}%
  \label{flc:non-wheat-sourdough}
\end{center}
\end{flowchart}

For non-wheat flours---including rye, emmer, and einkorn---no gluten
development has to be done, meaning there is no kneading, no
over-fermentation, and no issues with making flat bread.  In the case of rye
flour, sugars called pentosans prevent gluten bonds from properly
forming~\cite{rye+pentosans}.

\begin{figure}[!htb]
  \includegraphics[width=\textwidth]{final-bread}
  \caption[Sourdough rye bread]{A sourdough rye bread made using a loaf pan.
      The rye bread is not scored. The crust typically cracks open during
      baking.}%
  \label{fig:non-wheat-final-bread}
\end{figure}


This chapter will focus on making rye bread. The flour could
be replaced with einkorn or emmer based on your preference.

The following recipe will make you 2 loaves:

\begin{tabular}{r@{}rl@{}}
    \qty{1000}{\gram} &~(\qty{100}{\percent}) & Whole rye flour\\
    \qty{800}{\gram}  &  (\qty{80}{\percent}) & Water at room temperature\\
    \qty{200}{\gram}  &  (\qty{20}{\percent}) & Sourdough starter\\
    \qty{20}{\gram}   &   (\qty{2}{\percent}) & Salt\\
\end{tabular}

The sourdough starter can be in an active or inactive state. If it has been
at room temperature for a week with no feedings then it will be okay, same
if it has come right out of the fridge then still it will be no problem.
The dough is very forgiving.

If you follow the suggested quantities from the recipe you are making a
relatively wet rye dough. It's so wet that it can only be made using a loaf
pan. If you want to make a freestanding rye bread, consider reducing the
hydration to around~\qty{60}{\percent}.

\begin{figure}[!htb]
  \includegraphics[width=\textwidth]{ingredients}
  \caption[Non-wheat dough]{For non-wheat dough the ingredients are mixed
      together. There is no need to develop any dough strength. This
      simplifies the whole bread-making process.}%
  \label{fig:non-wheat-ingredients}
\end{figure}

Mix together all the ingredients with your hands, or opt for a spatula to
simplify things. Rye flour itself is very sticky and unpleasant to mix by
hand, the dough will stick a lot to your hands. If you use a stiff starter, it
could be easier to first dissolve it in the dough's water, then add the other
ingredients.

\begin{figure}[!htb]
  \includegraphics[width=\textwidth]{sticky-hands}
  \caption[Sticky rye dough]{Rye flour has a sugar molecule known as pentosan.
      These pentosans prevent the rye flour from building gluten bonds. As a
      result the dough never features an open crumb and is always very sticky
      when hand mixing.}%
  \label{fig:non-wheat-sticky-hands}
\end{figure}

The goal of the mixing process is simply to homogenize the dough, there
is no need to develop any dough strength. Once you see that
your sourdough starter has been properly incorporated, your
dough is ready to begin bulk fermentation.

You can bulk ferment the dough for a few hours up to
weeks. By extending the bulk fermentation time, you increase
the acidity the final loaf is going to feature. After around
48~hours, the acidity will no longer increase. This is because
most of the nutrients have been eaten by your microorganisms.
You could let your dough sit for longer, but it wouldn't alter the
final flavor profile by much.

I~recommend waiting until the dough has roughly increased
by~\qty{50}{\percent} in size. If you are daring, you can taste the dough to
get an idea of the acidity profile, it will likely taste very sour. However, a
lot of the acid will evaporate during the baking process, therefore the final
loaf will not be as sour as the dough you are tasting.

\begin{figure}[!htb]
  \includegraphics[width=\textwidth]{crumb}
  \caption[Rye bread]{The crumb structure of rye bread. By making a wetter
  dough, more water evaporates during the baking and thus the
  crumb tends to be a bit more open. Generally, rye
  bread is never as fluffy as wheat sourdough bread. The crust
  of this bread is a bit pale. The crust color can be controlled
  by baking the bread for a longer period.}%
  \label{fig:rye-crumb}
\end{figure}

Once you are happy with the acidity level, proceed to dividing
and shaping your dough.  If you made a drier dough, use as much
flour as needed to dry the dough a little bit and form a dough ball.
There is no folding the dough. All you do is tuck it together
as much as is needed to apply the shape of your banneton.

Shaping might not be possible if you opt for the wetter dough. Carefully spread
the dough with a spatula in your greased loaf pan, wetting the spatula to make
this process easier. Spread it until the surface looks smooth and shiny.

For proofing, I~recommend waiting around 60~minutes. An extended
proofing period does not make sense unless you want to further
increase the dough's acidity. The dough will not become fluffier
the longer you proof. With the short proofing period, however,
the dough will become a bit more homogeneous. This way the final
bread looks more uniform. The proofing period also allows the
dough to fully extend and fill the edges of the loaf pan. I~also
like to move the dough to the fridge for proofing. The dough stays
good in the fridge for weeks. You can proceed and bake it at a
convenient time for you. 

Once you are happy with the proofing stage, proceed and bake your dough
just like you'd normally do, more details can be found in
Chapter~\ref{chapter:baking}. One challenging aspect
of using a loaf pan is to make sure that the center part of your
dough is properly cooked. For this reason, it is best to use a thermometer
and measure the internal temperature. The bread is ready once the internal
temperature reaches \qty{92}{\degreeCelsius} (\qty{197}{\degF}). I~recommend
removing the bread from the loaf pan once it reaches the desired temperature,
then continue baking the loaf without the pan and steam. This way you achieve
a great crust all around your loaf, and can bake as long as you like until you
have achieved your crust color of choice. The darker, the more crunchy
the crust and the more flavor it offers. If you feel your dough might have
been overly acidic you can extend the baking time, as the longer you bake, the
more acidity will evaporate.

This is one of my favorite breads to bake which I~eat on an
almost daily basis. The effort required to make bread like
this is much lower compared to a wheat-based dough. In some
cases, I~extend the recipe and add additional sourdough discard
to the dough. You can add as much discard as you like. The resulting
bread will have a very complex but delicious flavor profile.
